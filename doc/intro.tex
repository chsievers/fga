%%%%%%%%%%%%%%%%%%%%%%%%%%%%%%%%%%%%%%%%%%%%%%%%%%%%%%%%%%%%%%%%%%%%%%%%%
%%
%W  intro.tex          FGA documentation                Christian Sievers
%%
%H  $Id$
%%
%Y  2003
%%

%%%%%%%%%%%%%%%%%%%%%%%%%%%%%%%%%%%%%%%%%%%%%%%%%%%%%%%%%%%%%%%%%%%%%%%%%
\Chapter{Overview of the FGA package}

This manual describes the {\FGA} (*Free Group Algorithms*) package,
a {\GAP} package for computations with finitely generated subgroups of
free groups.

This package allows you to test membership and conjugacy, and to compute
the rank ("RankOfFreeGroup"), the normalizer, centralizer, and index,
when the groups involved are finitely generated subgroups of free groups.

The methods which are used work mainly with finite automata.
Most of them are described in \cite{sims94}.
This implementation is documented in \cite{sievers03}.

In addition, it provides a finite presentation for the automorphism group
of a free group following \cite{neumann33}.

As this package mainly installs methods for already existing
{\GAP}-operations, there is not much to explain.
You find some new operations in chapter "The FGA package".

See Sections~"Installing the FGA Package"  and~"Loading  the  FGA
Package" for how to install and load the {\FGA} package.

If you are viewing this with on-line help, type: 

\beginexample
gap> ?>
\endexample

to see the functions provided by the {\FGA} package.

%%%%%%%%%%%%%%%%%%%%%%%%%%%%%%%%%%%%%%%%%%%%%%%%%%%%%%%%%%%%%%%%%%%%%%%%%
%%
%E  Emacs . . . . . . . . . . . . . . . . . . . . . local emacs variables
%%
%%  Local Variables:
%%  fill-column:    73
%%  End:
%%


