%%%%%%%%%%%%%%%%%%%%%%%%%%%%%%%%%%%%%%%%%%%%%%%%%%%%%%%%%%%%%%%%%%%%%%%%%
%%
%W  install.tex            GAP documentation            Christian Sievers
%%
%H  $Id$
%%
%Y  2003
%%

%%%%%%%%%%%%%%%%%%%%%%%%%%%%%%%%%%%%%%%%%%%%%%%%%%%%%%%%%%%%%%%%%%%%%%%%%
\Chapter{Installing and Loading the FGA Package}

%%%%%%%%%%%%%%%%%%%%%%%%%%%%%%%%%%%%%%%%%%%%%%%%%%%%%%%%%%%%%%%%%%%%%%%%%
\Section{Installing the FGA Package}

The installation of the {\FGA} package follows standard {\GAP} rules.
So the standard method is to copy the package into a directory `fga'
within the `pkg' directory of your {\GAP} distribution.  For other
non-standard options please see Chapter~"ref:Installing GAP Packages"
in the {\GAP} Reference Manual.

To create the documentation, go into the `doc' directory and type
`make_doc'.

%%%%%%%%%%%%%%%%%%%%%%%%%%%%%%%%%%%%%%%%%%%%%%%%%%%%%%%%%%%%%%%%%%%%%%%%%
\Section{Loading the FGA Package}

To use the {\FGA} Package you have to request it explicitly. This  is
done by calling

\beginexample
gap> LoadPackage("fga");
-----------------------------------------------------------------------------
Loading  FGA 0.9 (Free Group Algorithms)
by Christian Sievers (c.sievers@tu-bs.de).
-----------------------------------------------------------------------------
true
\endexample

The `LoadPackage' command is described in Section~"ref:LoadPackage"
in the {\GAP} Reference Manual.

%%%%%%%%%%%%%%%%%%%%%%%%%%%%%%%%%%%%%%%%%%%%%%%%%%%%%%%%%%%%%%%%%%%%%%%%%
%%
%E
